
\documentclass{article}

\usepackage{fullpage}
\usepackage[french]{babel} 
\usepackage[T1]{fontenc} 
\usepackage{amsmath} 
\usepackage{amsfonts}
\usepackage{mathabx}
\usepackage{graphicx}

\title{Bootstrap}

\begin{document}
\maketitle

\noindent Ci-dessous, je comprends du sujet.\\

\noindent On consid�re un n-�chantillon $\boldsymbol{X}^n = (X^n_1, ..., X^n_n)$ ind�pendantes et identiquement distribu�es suivant une loi dont on note $F_X$ la fonction de r�partition. On note le maximum $M_n = \max \{X_1, ..., X_n\}$.\\

\noindent Cette loi appartient au domaine d'attraction de Gumbel. Autrement dit, il existe une suite $(a_n,b_n)$ telle que 
\begin{equation}
\frac{M_n - b_n}{a_n} \overset{L}{\longrightarrow} G
\end{equation}
o� $G$ est la loi de Gumbel dont la fonction de r�partition $F_G$ s'�crit $\forall x \in \mathbb{R}$
\begin{equation}
F_G(x) = 1 - e^{-e^{-x}}
\end{equation}
Ce domaine est tr�s grand et de nombreuses lois classiques y appartiennent : exponentiel, gamma, logistique, log-normale, normale, ...\\

\noindent Dans cette situation, il a �t� montr� par de Haan qu'on peut choisir la suite $(a_n, b_n)$ comme suit
\begin{equation}
a_n = F_X^{-1}\left(1-\frac{1}{e n}\right) - F_X^{-1}\left(1- \frac{1}{n}\right) \qquad b_n = F_X^{-1}\left(1- \frac{1}{n}\right)
\end{equation}

\noindent L'objectif de l'�tude est de d�terminer la distribution de $M_n$ � partir du seul �chantillon $\boldsymbol{X}^n$. En d�signant par $P$ une loi quelconque, le param�tre d'int�r�t est donc $TP = P(M_n < x) = E_P[1(M_n < x)]$ $\forall x \in \mathbb{R}$.\\

\noindent La distribution (exacte) de $M_n$ s'�crit comme suit
\begin{equation}
T(P) = P(X_1 < x, ..., X_n < x) = [F_X(x)]^n
\end{equation}
Asymptotiquement, comme la loi appartient au domaine d'attraction de Gumbel
\begin{equation}
P\left(\frac{M_n - b_n}{a_n} < x\right) \simeq F_G(x) \quad \Longrightarrow \quad TP \simeq F_G(a_n x + b_n)
\end{equation}

\noindent Par commodit�, on consid�re donc aussi $\tilde{T}P = P\left(\frac{M_n - b_n}{a_n} < x\right)$.\\

\noindent La distribution bootstrap d'Efron de $M_n$ s'�crit comme suit
\begin{equation}
P^*_n = \left[1-\left(\frac{n-1}{n}\right)^n\right] \delta_{X_{(n)}} + ... +  \left[\left(\frac{k}{n}\right)^n-\left(\frac{k-1}{n}\right)^n\right] \delta_{X_{(k)}} + ... + \left(\frac{1}{n}\right)^n  \delta_{X_{(1)}}
\end{equation}
Fukuchi a montr� qu'asymptotiquement cette loi convergeait vers un processus stochastique fonction du tirage effectu�. Faut-il d�velopper ce point ou utiliser un autre argument ?\\

\noindent Si on tire seulement $m$ avec $m < n$, on a
\begin{equation}
P^*_{m|n} = \left[1-\left(\frac{n-1}{n}\right)^m\right] \delta_{X_{(n)}} + ... +  \left[\left(\frac{k}{n}\right)^m-\left(\frac{k-1}{n}\right)^m\right] \delta_{X_{(k)}} + ... + \left(\frac{1}{n}\right)^m  \delta_{X_{(1)}}
\end{equation}
Fukuchi montre que $m = o(n)$, $\tilde{T}P^*_{m|n}$ converge vers $F_G$ avec la distance  de Kolmogorov. Comme on ne conna�t pas $a_n$ et $b_n$, on peut l'estimer comme suit et travailler par analogie.
\begin{equation}
\hat{a}_n = \hat{F}_X^{-1}\left(1-\frac{1}{e n}\right) - \hat{F}_X^{-1}\left(1- \frac{1}{n}\right) \qquad \hat{b}_n = \hat{F}_X^{-1}\left(1- \frac{1}{n}\right)
\end{equation}
o� $\hat{F}_X$ est la distribution empirique construite � partir de l'�chantillon $\boldsymbol{X}^n$. C'est la version m out of n qui a priori fonctionne pour la distribution de $M_n$. \\

\noindent Je ne vois pas la diff�rence entre le bootstrap sous-�chantillonn� et le m out of n. Sais-tu quelle est elle ?\\

\noindent A priori, la vitesse de convergence est donn�e par $a_n$. On peut alors faire une r�gression pour trouver $a_n$ sous la forme $n^\alpha$. Il y a un article de Bertail sur le sujet : on subsampling estimators with unknown rate of convergence. En regardant rapidement, l'article donne les preuves que �a fonctionne sous condition de choisir les points o� on regarde l'�volution de la courbe lorsque $m$ varie.\\

\noindent La fonction de r�partition $\tilde{T}P^*_{m|n}$ est en escalier. On pourrait donc acc�l�rer la convergence en la lissant par exemple � l'aide d'un noyau.

\end{document}

